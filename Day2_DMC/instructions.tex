\documentclass[12pt]{article}
\usepackage{helvet}
\usepackage{fullpage}
\usepackage{amsmath}
\begin{document}
\begin{center}
	{\bf \large Diffusion Monte Carlo } 
\end{center} 

\section*{Setting up your directory} 

You will need the following working files from the Day1 VMC exercise: metropolis.py, slaterwf.py, wavefunction.py, and hamiltonian.py. 
You can copy the files into your working directory, or you can put 
\begin{verbatim}
	import sys
	sys.path.append("/path/to/Day1_VMC/")
\end{verbatim}
at the beginning of your Python file.


\section*{The trajectories.} 

We're going to write a series of implementations of diffusion Monte Carlo.
Each implementation will generate data with the following columns: 

\begin{itemize}
	\item tau
	\item step
	\item elocal (local energy averaged over all configurations)
	\item weight (averaged over all configurations)
	\item elocalvar (local energy variance)
	\item weightvar (weight variance)
	\item eref (reference energy)
\end{itemize}

We will do our testing for $\tau\simeq 0.01$ and using $\alpha=2,\beta=0.5$ in the Slater-Jastrow wave function.
We do this because these values satisfy the cusp conditions and so there will not be any divergences in the local energy.

	
\section*{Importance sampling} 

For 1000 walkers, generate dynamics as follows: 
\begin{equation}
x_{n+1} = x_n + \sqrt{\tau}\chi + \tau \frac{\nabla \Psi_T(x_n)}{\Psi_T(x_n)},
\end{equation}
where $\chi$ is a random variable. 
You can add an acceptance/rejection step to this just like in VMC, although the reasoning is quite different.
Set all weights to one.
Make a CSV file and use Pandas to analyze your data. 

Remember:
\begin{equation}
	a(x' \leftarrow x) = \text{min}\left(1,\frac{\Psi_T^2(x') T(x \leftarrow x')}{\Psi_T^2(x) T(x' \leftarrow x)} \right)
\end{equation}

\begin{equation}
	T(x'\leftarrow x) = \exp\left(-\frac{ \left(x' - x - \tau \frac{\nabla \Psi_T(x)}{\Psi_T(x)}\right)^2}{ 2\tau}\right)
\end{equation}

\begin{itemize}
\item You should be able to get the VMC result for your trial wave function. Does it match? 
\end{itemize}

\section*{Importance sampling with weights}
Now we will update the weights $w_i$. Set 
\begin{equation}
w*=	\exp[-\tau E_L(x_{n+1})] 
\end{equation}
each step.
You probably don't need to perform this for too many steps!

\begin{itemize}
\item Track the weights of the walkers and their values as a function of step. What happens?
\end{itemize}
To check this, use the \texttt{dmc\_plot.py} file.


\section*{Fixing the normalization}

We want the weights to average around 1.
Use a shift 
\begin{equation}
w*=	\exp[-\tau (E_L(x_{n+1})-E_{\text{ref}})] 
\end{equation}
We can adjust $E_{\text {ref}}$ to ensure the weights average to one:
\begin{equation}
E_{\text{ref}}=E_{\text{ref}}-\log(\langle w \rangle) 
\end{equation}
where $\langle w \rangle$ is the average weight. 

\begin{itemize}
\item Now plot the weights. They should average to 1 but what happens to the variance?
\end{itemize}
To check this, use the \texttt{dmc\_plot.py} file.



\section*{Branching} 

We now want to split the walkers with too-large weights and kill the walkers with too-small weights. 
There are many ways to do this; we will choose one that lets us keep the number of walkers constant. 
Every step:
\begin{enumerate}
\item Stack up weights (np.cumsum)
\item Throw random numbers for each walker (np.random.random)
\item Copy the walker corresponding to where each random number landed (np.searchsorted)
\item Assign new walkers (slicing)
\item Set all weights to the average weight.
\end{enumerate}
\section*{Using the algorithm}
Congratulations! We have now implemented the DMC algorithm. 
Let's check some things:


\begin{itemize}
\item How does the variance of the weights behave now? Is it more stable?
\item The exact energy is -2.903724. Do we get that in the $\tau\rightarrow0$ limit?
\end{itemize}

To check this, use the \texttt{dmc\_reblock.py} file.

\section*{More things to do}
\begin{itemize}
\item As you change the number of walkers, how does the behavior of $E_{\text ref}$ change?
\item Implement a triplet wave function and fixed node. What is the node of this wave function?
\item What might you change to improve the timestep error? 
\end{itemize}


\end{document}